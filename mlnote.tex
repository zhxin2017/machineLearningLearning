\documentclass[UTF8]{ctexart}
\title{机器学习笔记}
\author{张鑫}
\date{}
\usepackage{amsmath}
\DeclareMathOperator{\tr}{tr}
\newtheorem{theorem}{定理}[section]
\begin{document}
	
	\maketitle

	\section{概论}
	\subsection{统计学习三要素}
	统计学习三要素:模型、策略、算法. 在监督学习过程中,模型就是所要学习的条件概率分布或决策函数. 模型的假设空间包含所有可能的条件概率分布或决策函数. 统计学习的目标在于从假设空间中选取最优模型. 监督学习的两个基本策略:经验风险最小化和结构风险最小化. 统计学习的算法为求解最优化问题的算法.

	\subsection{模型评估与模型选择}
	
	\subsection{机器学习的可能性}
	\subsubsection{Hoeffding不等式}
	
	\section{线性回归}
	
	\subsection{Cost Function}
	\begin{equation*}
		h_\theta(x)=\theta^Tx
	\end{equation*}

	Assume: 
	\begin{equation*}
		y^{(i)}=\theta^Tx^{(i)}+\epsilon^{(i)}
	\end{equation*}
	where $\epsilon^{(i)}$ is an error term. Further assume $\epsilon^{(i)}$ is i.i.d. and with a Gaussian distribution of mean 0 and variance $\sigma^2$:
	\begin{equation*}
		p(\epsilon^{(i)}) = \frac{1}{\sqrt{2\pi}\sigma}e^{-\frac{(\epsilon^{(i)})^2}{2\sigma^2}}
	\end{equation*}
	This implies that
	\begin{equation*}
		p(y^{(i)}|x^{(i)};\theta) = \frac{1}{\sqrt{2\pi}\sigma}e^{-\frac{(y^{(i)}-\theta^Tx^{(i)})^2}{2\sigma^2}}
	\end{equation*}
	Likelihood function:
	\begin{equation*}
	\begin{aligned}
		L(\theta) & = L(\theta;X,y) = p(y|X;\theta) \\
		& = \prod_{i=1}^{m}p(y^{(i)}|x^{(i)};\theta) \\
		& = \prod_{i=1}^{m}\frac{1}{\sqrt{2\pi}\sigma}e^{-\frac{(y^{(i)}-\theta^Tx^{(i)})^2}{2\sigma^2}}
	\end{aligned}
	\end{equation*}
	Log likelihood:
	\begin{equation*}
	\begin{aligned}
		l(\theta) & = log(L(\theta)) \\
		& = log(\prod_{i=1}^{m}\frac{1}{\sqrt{2\pi}\sigma}e^{-\frac{(y^{(i)}-\theta^Tx^{(i)})^2}{2\sigma^2}}) \\
		& = \sum_{i=1}^{m}log(\frac{1}{\sqrt{2\pi}\sigma}e^{-\frac{(y^{(i)}-\theta^Tx^{(i)})^2}{2\sigma^2}})
	\end{aligned}
	\end{equation*}
	对数似然函数最大化等价于最小化如下 cost function:
	\begin{equation*}
		J(\theta)=\frac{1}{2}\sum_{i=1}^{m}(y^{(i)}-h_\theta(x^{(i)}))^2
	\end{equation*}
	\subsection{LMS algorithm}
	只有一个训练样本$(x,y)$时的导数:
	\begin{equation*}
	\begin{aligned}
	\frac{\partial}{\partial\theta_j}J(\theta) & =\frac{\partial}{\partial\theta_j}\frac{1}{2}(y-h_\theta(x))^2 \\
	& =(h_\theta(x)-y)\frac{\partial}{\partial\theta_j}(\sum_{i=0}^{n}\theta_ix_i-y)\\
	& =(h_\theta(x)-y)x_j
	\end{aligned}
	\end{equation*}
	相应的更新规则:
	\begin{equation*}
		\theta_j:=\theta_j-\alpha(h_\theta(x^{(i)})-y^i)x^{(i)}_j
	\end{equation*}
	多个样本的更新规则(Batch gradient descent):
	\begin{equation*}
			\theta_j:=\theta_j-\alpha\sum_{i=0}^{n}(h_\theta(x^{(i)})-y^i)x^{(i)}_j
	\end{equation*}
	
	\subsection{Matrix derivatives}
	迹技巧:
	If $A$ and $B$ are square matrices and $a$ is a real number,
	\begin{equation*}
	\begin{aligned}
	\tr AB&=\tr BA \\
	\tr A&=\tr A^T \\
	\tr (A+B)&=\tr A + \tr B \\
	\tr aA &= a\tr A 
	\end{aligned}	
	\end{equation*}
	
	Matrix derivatives:
	\begin{equation*}
	\begin{aligned}
	\nabla_A\tr AB & =B^T\\
	\nabla_{A^T}f(A) & =(\nabla_Af(A))^T\\
	\nabla_A\tr ABA^TC & =CAB+C^TAB^T\\
	\nabla_A{|A|}&=|A|(A^{-1})^T
	\end{aligned}	
	\end{equation*}
	
	Normal equations:
	\begin{equation*}
		\theta = (X^TX)^{-1}X^Ty
	\end{equation*}
	
	\section{Logistic Regression}
	\begin{equation*}
		h_\theta(x)=g(\theta^Tx)=\frac{1}{1+e^{-\theta^Tx}},
	\end{equation*}
	where
	\begin{equation*}
		g(z)=\frac{1}{1+e^{-z}}
	\end{equation*}
	\begin{equation*}
		g'(z)=g(z)(1-g(z))
	\end{equation*}
	Assume that	
	\begin{equation*}
	\begin{aligned}
	P(y=1|x;\theta) & =h_\theta(x) \\
	P(y=0|x;\theta) & =1-h_\theta(x)
	\end{aligned}
	\end{equation*}
	等同于
	\begin{equation*}
		p(y|x;\theta)=(h_\theta(x))^y(1-h_\theta(x))^{(1-y)}
	\end{equation*}
	Assuming that the m training examples were generated independently, then the likelihood of the parameters is
	\begin{equation*}
		L(\theta) = \prod_{i=i}^{m}(h_\theta(x^{(i)}))^{y^{(i)}}(1-h_\theta(x^{(i)}))^{(1-y^{(i)})}
	\end{equation*}
	The log likelihood is
	\begin{equation*}
		l(\theta) = \sum_{i=i}^{m}\log((h_\theta(x^{(i)}))^{y^{(i)}}(1-h_\theta(x^{(i)}))^{(1-y^{(i)})})
	\end{equation*}
	定义损失函数
	\begin{equation*}
	\begin{aligned}
		J(\theta) &=-l(\theta) \\
		& = \sum_{i=i}^{m}-\log((h_\theta(x^{(i)}))^{y^{(i)}}(1-h_\theta(x^{(i)}))^{(1-y^{(i)})}) \\
		& = \sum_{i=i}^{m}-y^{(i)}\log((h_\theta(x^{(i)}))-(1-y^{(i)})\log(1-h_\theta(x^{(i)}))
	\end{aligned}
	\end{equation*}
	导数为
	\begin{equation*}
	\begin{aligned}
	\frac{\partial J(\theta)}{\partial \theta_j} & = \sum_{i=1}^{m}-y^{(i)}\frac{h_\theta(x^{(i)})(1-h_\theta(x^{(i)}))}{h_\theta(x^{(i)})}x_j^{(i)}-(1-y^{(i)})\frac{-h_\theta(x^{(i)})(1-h_\theta(x^{(i)}))}{1-h_\theta(x^{(i)})}x_j^{(i)} \\
	& =\sum_{i=1}^{m}(-y^{(i)}(1-h_\theta(x^{(i)}))+(1-y^{(i)})h_\theta(x^{(i)}))x_j^{(i)} \\
	& =\sum_{i=1}^{m}(h_\theta(x^{(i)})-y^{(i)})x_j^{(i)}
	\end{aligned}
	\end{equation*}
	多个样本的更新规则(\emph{和线性回归更新规则相同}):
	\begin{equation*}
	\theta_j:=\theta_j-\alpha\sum_{i=0}^{n}(h_\theta(x^{(i)})-y^i)x^{(i)}_j
	\end{equation*}
	
	
	\section{广义线性模型}
	The exponential family:
	\begin{equation*}
		p(y;\eta)=b(y)\exp(\eta^TT(y)-a(\eta))
	\end{equation*}
	$\eta$ is the \emph{natural parameter}(or \emph{canonical parameter}); $T(y)$ is the \emph{sufficient statistic}; $a(\eta)$ is the \emph{log partition function}.
	
	
	\section{感知机}
	\subsection{感知机学习策略}

	感知机是一种线性分类模型,属于判别模型。
	
	假设训练数据集线性可分,输入空间$R^n$中任一点到超平面$S$的距离为:
	\[
		-\frac{1}{\|w\|}y_i(w^T\cdot{x_i}+b)
	\]
	
	假设超平面$S$的误分类点所有误分类点到超平面$S$的总距离为:
	\[
		-\frac{1}{\|w\|}\sum_{x_i\in{M}}y_i(w^T\cdot{x_i}+b)
	\]
	
	不考虑$\frac{1}{\|w\|}$,得到感知机学习的损失函数:
	\[
		L(w,b)=-\sum_{x_i\in{M}}y_i(w^T\cdot{x_i}+b)
	\]
	
	\subsection{感知机学习算法}
	\subsubsection{算法的收敛性}
	\begin{theorem}
		设训练数据集$T=\{(x_1,y_1),(x_2,y_2),\dots,(x_N,y_N)\}$是线性可分的,则
		
		(1)存在满足条件$\|\hat{w}_{opt}\|=1$的超平面将T完全正确分开;且存在$\gamma>0$,对所有$i=1,2,\dots,N$
		\[
			y_i(\hat{w}_{opt}\cdot\hat{x}_i)=y_i(w_{opt}\cdot x_i+b_{opt})\geq\gamma
		\]
		
		(2) 令$R=\max\limits_{1\leq i \leq N}\|\hat{x}_i\|$,则感知机算法在$T$上的误分类次数$k$满足不等式
		\[
			k\leq(\frac{R}{\gamma})^2
		\]
	\end{theorem}

	\section{Support Vector Machine}
	\begin{equation*}
		h_{w, b}(x) = g(w^Tx + b).
	\end{equation*}
	Here, $g(z)=1$ if $z>0$, and $g(z)=-1$ otherwise.
	
	Functional margin with respect to the training example $(x^{(i)}, y^{(i)})$:
	\begin{equation*}
		\hat{\gamma}^{(i)} = y^{(i)}(w^Tx^{(i)} + b)
	\end{equation*}
	
	Define the function margin of $(w, b)$ with respect to training set to be the smallest of the functional margins of the individual training examples:
	\begin{equation*}
		\hat{\gamma} = \min_{i=1,\cdots,m}\hat{\gamma}^{(i)}
	\end{equation*}
	
	
\end{document}